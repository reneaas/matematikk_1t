\documentclass[10pt]{article}
\usepackage[usenames]{color} %used for font color
\usepackage{amssymb} %maths
\usepackage{amsmath} %maths
\usepackage[utf8]{inputenc} %useful to type directly diacritic characters
\usepackage{polynom}
\usepackage{annotate-equations}
\begin{document}

\begin{equation*}
 \eqnmarkbox[black]{node1}{f(x)} = \eqnmarkbox[blue]{node2}{a}(x - \eqnmarkbox[red]{node3}{x_1}) (x - \eqnmarkbox[red]{node4}{x_2})
\end{equation*}

\annotate[yshift=1em]{left}{node1}{Funksjonsverdi}

\annotate[yshift=-1em]{below, left}{node2}{Ledende koeffisient}

\annotatetwo[yshift=1em]{above}{node3}{node4}{Nullpunkter}

\end{document}