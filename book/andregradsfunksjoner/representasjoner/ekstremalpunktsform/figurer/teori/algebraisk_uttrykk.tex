\documentclass[10pt]{article}
\usepackage[usenames]{color} %used for font color
\usepackage{amssymb} %maths
\usepackage{amsmath} %maths
\usepackage[utf8]{inputenc} %useful to type directly diacritic characters
\usepackage{longdivision}
\usepackage{annotate-equations}
\begin{document}
\begin{equation*}
 \eqnmarkbox[black]{node1}{f(x)} = \eqnmarkbox[blue]{node2}{a}(x - \eqnmarkbox[red]{node3}{x_0})^2 + \eqnmarkbox[red]{node4}{y_0}
\end{equation*}

\annotate[yshift=1em]{left}{node1}{Funksjonsverdi}

\annotate[yshift=-1em]{below, left}{node2}{Ledende koeffisient}

\annotate[yshift=1em]{above, right}{node3, node4}{Ekstremalpunkt $(x_0, y_0)$}
\annotate[yshift=-1em]{below}{node3}{Symmetrilinje}

\end{document}