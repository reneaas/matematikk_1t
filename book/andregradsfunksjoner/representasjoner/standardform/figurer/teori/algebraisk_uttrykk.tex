\documentclass[10pt]{article}
\usepackage[usenames]{color} %used for font color
\usepackage{amssymb} %maths
\usepackage{amsmath} %maths
\usepackage[utf8]{inputenc} %useful to type directly diacritic characters
\usepackage{polynom}
\usepackage{annotate-equations}
\begin{document}
\begin{equation*}
  \eqnmarkbox[black]{node1}{f(x)} = 
  \eqnmarkbox[blue]{node2}{a} x^2
+ \eqnmarkbox[red]{node3}{b} x
+ \eqnmarkbox[teal]{node4}{c} 
\end{equation*}

\annotate[yshift=1em]{left}{node1}{Funksjonsverdi}
\annotate[yshift=1em]{}{node2, node3, node4}{Koeffisienter}
\annotate[yshift=-1em]{below}{node2}{Ledende koeffisient}
\end{document}